% !TeX program = xelatex

% UI-Thesis v1.0

% این قالب بر اساس فرمت پایان‌نامه‌ها و رساله‌های تحصیلات تکمیلی دانشگاه اصفهان تهیه شده است.
% علیرضا روحی-دانشجوی دکتری گروه مهندسی نرم افزار دانشگاه اصفهان
% 1395
% rouhi.ir@gmail.com
% توصیه می‌شود که از توزیع تک‌لایو (TexLive2015) به بعد استفاده شود:
% http://tug.org/texlive/acquire-iso.html

% موفق باشید.
% با تشکر از امین فخاری که قالب اصلی این پایان نامه را برای دانشگاه صنعتی اصفهان تهیه نموده اند.
% 1395
% a101.fakhari@gmail.com
% -----------------------------------------------------------------------------------

% نکات:

% برای آن‌که پردازش فایل و مشاهده خروجی در هنگام نوشتن پایان‌نامه آسان‌تر و سریع‌تر انجام شود، انجام موارد زیر توصیه می گردد:
% الف) فصل‌ها و بخش‌هایی که در حال نوشتن آن‌ها نیستید را غیر فعال کنید. به‌عنوان مثال، در این قالب، این دستورات را می‌توان در صورت عدم نیاز با اضافه کردن % به طور موقت غیرفعال کرد:
% \MakeTitlePage
% \MakeFarsiSignaturePage
% \input{Chapters/Acknowledgments}
% \MakeCopyRightPage
% \input{Chapters/Dedication}
% \MakeTableOfContents
% \MakeListOfFigures
% \MakeListOfTables
% \MakeFarsiAbstract
% \input{Chapters/Chapter#}
% \MakeAppendices
% \input{Chapters/Appendices}
% \MakeEnglishAbstract
% \MakeEnglishSignaturePage
% ب) از گزینه draft برای فراخوانی کلاس استفاده کنید. یعنی
% \documentclass[a4paper,fleqn,13pt,twoside,draft]{book}
% این گزینه حالت چرکنویس را ایفا می‌کند و بر روی بسته‌های مختلف اثرهای متفاوتی دارد. به‌عنوان مثال: به جای شکل، تنها چهارچوب آن نمایش داده شود، لینک‌های hyperref غیر فعال گردد، فایل‌های خارجی را در بسته listings اضافه نمی‌کند و ... و همه این موارد سبب کاهش زمان اجرا و حجم فایل می‌شود.

% در صورتی که میخواهید به سطر بعد بروید اما نمیخواهید بین دو کلمه‌ای که نوشتید فاصله بیفتد کافی است در انتهای خط اول  (بدون فاصله) کاراکتر % را اضافه کنید. با این عمل، لاتک خط فاصله ایجاد شده در اثر تغییر سطر را به عنوان توضیح اضافه یا کامنت در نظر میگیرد و در خروجی اعمال نمی‌کند.

% توصیه می‌شود از شکل‌های برداری با فرمت PDF استفاده شود. این کار علاوه بر افزایش کیفیت رسال/پایان‌نامه/گزارش، باعث کاهش حجم شکل‌ها (و در نتیجه  کاهش حجم فایل نهایی) و همچنین کاهش زمان پردازش می‌شود.

% در این قالب سعی شده است که از تمامی بخش‌های موجود در پایان‌نامه‌ها نمونه‌ای آورده شود.

\documentclass[a4paper,fleqn,11pt,oneside]{book}
\usepackage{tikz}
\usetikzlibrary{arrows,shapes}
\usepackage{tabulary}
\usepackage{Settings/UI-Thesis}
\usepackage{acronym}
\usepackage{multirow}

%-----------------------------
% دستورهای مورد نیاز را در این قسمت اضافه نمایید:
% Cross-reference commands.
\newtheorem{thm}{Theorem}[chapter]
\newtheorem{asmp}{Theorem}[chapter]
\theoremstyle{definition}
\newtheorem{definition}[thm]{تعریف}
\newtheorem{assumption}[asmp]{فرض}
\newtheorem{example}{مثال}[chapter]

\newcommand{\xs}[1]{بخش~\ref{#1}}
\newcommand{\xc}[1]{فصل~\ref{#1}}
\newcommand{\xp}[1]{صفحه~\pageref{#1}}
\newcommand{\xf}[1]{شکل~\ref{#1}}
\newcommand{\xt}[1]{جدول~\ref{#1}}
\newcommand{\xa}[1]{پیوست~\ref{#1}}
\newcommand{\xd}[1]{تعریف~\ref{#1}}
\newcommand{\xr}[1]{قانون~\ref{#1}}
\newcommand{\xra}[1]{R~\ref{#1}}
\newcommand{\xl}[1]{کد~\ref{#1}}
\newcommand{\xal}[1]{الگوریتم~\ref{#1}}
\newcommand{\xe}[1]{معادله~\eqref{#1}}
\newcommand{\xex}[1]{مثال~\ref{#1}}
\newcommand{\xeq}[1]{رابطه~\eqref{#1}}

\newcommand{\mylr}[1]{\texorpdfstring{\lr{#1}}{#1}}
\eqenvironment{نکات}{itemize}
\eqenvironment{تعریف}{definition}
\eqcommand{مورد}{item}

%%%%%%%%%%%%%%%%%%%%%%%%%%%%%%%%%%%%%%%%%%%%%%%%%%%%%%%%%%%%%%



%\DeclareMathSizes{9}{9}{9}{9}

\lstset{escapeinside={/*@}{@*/}}
%\definecolor{codebackground}{rgb}{0.95,0.95,0.95}
\definecolor{codebackground}{RGB}{255,255,255}
\definecolor{commentcolor}{RGB}{77,153,153}
\definecolor{keywordcolor}{RGB}{153,77,153}
\lstset{backgroundcolor=\color{codebackground}}

\lstset{
  captionpos=b,
  numberstyle=\tiny,
  %basicstyle=\ttfamily\footnotesize,
  %basicstyle=\setLTR\thefootnotesize\ttfamily,
  basicstyle=\setLTR\bfseries\fontsize{8.5pt}{0}\selectfont\ttfamily,
  columns=flexible,
  tabsize=2,
  numbers=none, %left,
  nolol=true,
  keywordstyle=\color{keywordcolor},
  commentstyle=\color{commentcolor},
  stringstyle=\color{blue},
  captiondirection=RTL,
  upquote=true,
}

\def\lstlistingname{کد}

%\input{listings/EOLFormat}
%-----------------------------

\begin{document}



\pagestyle{plain}
\pagenumbering{harfi}
%\setcounter{page}{2}

% ░░░░░░░▒▒▒▒▒▒▓▓▓▓ In the Name of Allah ▓▓▓▓▒▒▒▒▒▒░░░░░░░
\clearpage
\thispagestyle{empty}
\newgeometry{left=3.5cm, right=3.5cm, top=3.5cm, bottom=3.5cm}
\begin{figure}
  \centering
\includegraphics[width = \linewidth]{Settings/Allah.png}
\end{figure}
% ░░░░░░░▒▒▒▒▒▒▓▓▓▓ Title Page ▓▓▓▓▒▒▒▒▒▒░░░░░░░
\DepartmentFa{دانشکده مهندسی کامپیوتر}
\GroupFa{گروه مهندسی نرم‌افزار}
\ThesisTypeFa{پروژه درس روش پژوهش و ارائه} % Or \ThesisTypeFa{پایان‌نامه} Or \ThesisTypeFa{پیشنهادیه پایان‌نامه}
\DegreeFa{کارشناسی} % Or \DegreeFa{کارشناسی ارشد}
\FieldFa{مهندسی کامپیوتر}
\BranchFa{ هوش مصنوعی}   % This is نام گرایش
\YourFullnameFa{ سهیل سلیمی}
\FirstSupervisorFa{زهرا زجاجی}
\YearFa{ دی 1402}
\TitleFa{ استفاده از کانتینر ها برای اجرای سریع و محلی\\ [0.4cm] مدل های زبانی بزرگ}

% اگر عنوان رساله طولانی بود، در دو خط به صورت نشان داده شده تقسیم شود.
%\TitleFa{قسمت اول عنوان \\ [0.4cm] قسمت دوم عنوان}

\MakeTitlePage


% ░░░░░░░▒▒▒▒▒▒▓▓▓▓ Dedication ▓▓▓▓▒▒▒▒▒▒░░░░░░░
\input{Chapters/Dedication}

% ░░░░░░░▒▒▒▒▒▒▓▓▓▓ Abstract - Farsi ▓▓▓▓▒▒▒▒▒▒░░░░░░░
% !TEX root = ../ui-thesis.tex
% !TeX program = xelatex


\AbstractFa{
  مدل های زبانی بزرگ \LTRfootnote{Large language model} مدل هایی هستند که با استفاده از تکنیک های یادگیری عمیق بر روی داده های متنی بزرگ آموزش دیده اند و قادر به تولید متن های شبیه به انسان و انجام وظایف مختلف بر اساس ورودی ارائه شده هستند. این مدل ها می توانند برای تولید محتوای خلاق، ترجمه زبان ها، پاسخ به سوالات و انجام وظایف دیگر مورد استفاده قرار گیرند. اما اجرای این مدل ها در محیط های واقعی با چالش هایی مانند نیاز به منابع محاسباتی زیاد، حفظ حریم خصوصی و امنیت داده ها، و مسئولیت اخلاقی مواجه است. در این مقاله، ما یک روش برای استفاده از کانتینر ها برای اجرای سریع و محلی ﻣمدل ﻫﺎﻱ ﺯﺑﺎﻧﻲ ﺑﺰﺭگ را ارائه می دهیم. کانتینر ها امکان ایجاد و اجرای محیط های نرم افزاری مستقل و قابل حمل را فراهم می کنند. ما نشان می دهیم که چگونه می توان با استفاده از کانتینر ها، مدل های زبانی را بدون نیاز به یک سرویس ابری، بر روی دستگاه های محلی اجرا کرد. ما مزایا و چالش های این روش را بررسی می کنیم و چندین مورد کاربردی را نشان می دهیم. ما نتایج آزمایش های خود را بر روی چندین مدل زبانی معروف و چندین وظیفه زبانی ارائه می دهیم و نشان می دهیم که این روش می تواند کارایی و دقت بالایی را حفظ کند. ما همچنین چندین جهت برای کارهای آینده در این زمینه پیشنهاد می دهیم.  
}

\KeywordsFa{
 1- کانتینر ها و مدل های زبانی بزرگ
 2- اجرای محلی و سریع مدل های زبانی
 3- بهینه سازی و امنیت مدل های زبانی
 4- مورد کاربردی مدل های زبانی بزرگ
}
\MakeFarsiAbstract

% ░░░░░░░▒▒▒▒▒▒▓▓▓▓ Table of Contents/Figures/Tables ▓▓▓▓▒▒▒▒▒▒░░░░░░░
\setcounter{page}{0}

\MakeTableOfContents
\MakeListOfFigures
\MakeListOfTables

% ----------------------------------------------------------------------------
\clearpage
\pagestyle{myheadings}
\pagenumbering{arabic}
\setcounter{page}{1}

% ░░░░░░░▒▒▒▒▒▒▓▓▓▓ Chapters ▓▓▓▓▒▒▒▒▒▒░░░░░░░
\clearpage
\baselineskip=0.9cm
\setcounter{footnote}{0}

% !TEX root = ../ui-thesis.tex
% !TeX program = xelatex
\chapter{مقدمه}
\section{پیش‌گفتار}

مدل های زبانی بزرگ \LTRfootnote{Large language model} در سال های اخیر توانایی شگفت انگیزی در وظایف پردازش زبان طبیعی و فراتر از آن نشان داده اند. این موفقیت مدل های زبانی بزرگ  ها منجر به ورود تعداد زیادی از پژوهش های علمی در این زمینه شده است. این پژوهش ها موضوعات متنوعی را شامل می شوند، از جمله نوآوری های معماری، راهبردهای بهتر آموزش، بهبود طول متن، تنظیم دقیق، مدل های زبانی بزرگ  های چند حالته، رباتیک، مجموعه داده ها، معیارهای ارزیابی، کارایی و بیشتر. با توجه به توسعه سریع روش ها و پیشرفت های مداوم در پژوهش مدل های زبانی بزرگ  ها، درک تصویر کلی از پیشرفت ها در این راستا بسیار چالش برانگیز شده است.

در این مقاله، ما به بررسی چالش ها و راه حل های مربوط به اجرای سریع و محلی مدل های زبانی بزرگ  ها می پردازیم. ما نشان می دهیم که چگونه می توان با استفاده از کانتینر ها، یک روش مدیریت بسته بندی و انتشار نرم افزار، مدل های زبانی بزرگ  ها را بر روی رایانه های شخصی یا سرورهای خود اجرا کرد. این به معنای این است که نیازی به تکیه بر یک سرویس ابری برای استفاده از آنها نیست، که می تواند چندین مزیت داشته باشد، از جمله: حفظ حریم خصوصی و امنیت داده ها: وقتی یک مدل های زبانی بزرگ  محلی را اجرا می کنید، داده های شما هرگز از دستگاه شما خارج نمی شود. این می تواند برای داده های حساس، مانند سوابق بهداشتی یا داده های مالی، مهم باشد. در دسترس بودن آفلاین: مدل های زبانی بزرگ  های محلی می توانند آفلاین استفاده شوند، که به این معنی است که شما می توانید از آنها حتی اگر اتصال اینترنت نداشته باشید، استفاده کنید. این می تواند برای کار بر روی پروژه های در مناطق دور افتاده یا برای برنامه هایی که نیاز به در دسترس بودن همیشگی دارند، مفید باشد. سفارشی سازی: مدل های زبانی بزرگ  های محلی می توانند برای وظایف یا حوزه های خاص تنظیم دقیق شوند. این می تواند آنها را دقیق تر و کارآمدتر برای وظایفی که شما نیاز دارید انجام دهند، کند. مدل های زبانی بزرگ  ها می توانند بر روی انواع پلتفرم های سخت افزاری، از جمله واحد پردازش مرکزی \LTRfootnote{CPU} ها و واحد پردازش گرافیکی \LTRfootnote{GPU} ها اجرا شوند. با این حال، مهم است توجه داشته باشید که مدل های زبانی بزرگ  های محلی می توانند بسیار هزینه بر برای اجرا باشند، بنابراین شما ممکن است به یک رایانه قدرتمند برای استفاده از آنها به طور موثر نیاز داشته باشید. برای اجرای یک مدل های زبانی بزرگ  محلی، شما باید نرم افزار لازم را نصب کنید و فایل های مدل را دانلود کنید. پس از انجام این کار، شما می توانید مدل را شروع کنید و از آن برای تولید متن، ترجمه زبان ها، پاسخ به سوالات و انجام وظایف دیگر استفاده کنید. 
% !TEX root = ../ui-thesis.tex
% !TeX program = xelatex

\chapter{مطالب اصلی}
\section{ادبیات موضوع}
\subsection{کانتینر ها}

کانتینر سازی یک روش مدیریت بسته بندی و انتشار نرم افزار است که به شکل مجازی برنامه های کاربردی را برای استقرار، بسته بندی و ایزوله می کند1. کانتینرها از هسته سیستم عامل استفاده می کنند تا برنامه های کاربردی را از سخت افزار و سیستم عامل میزبان جدا کنند. این باعث می شود که برنامه ها قابل انتقال، سبک و سریع باشند2.

مدیریت کانتینر یک فرآیند است که به ایجاد، اجرا، مانیتورینگ، توقف و حذف کانتینرها می پردازد3. برای مدیریت کانتینرها، نیاز به ابزارهایی است که به عنوان ارکستراسیون کانتینر شناخته می شوند. این ابزارها به مدیریت کانتینرهای متعدد بر روی یک یا چند سرور کمک می کنند. برخی از این ابزارها عبارتند از:
\begin{itemize}[label=-]
\item
\lr{Docker}: این ابزار یک پلتفرم کانتینر سازی است که به ساخت، اجرا و اشتراک گذاری کانتینرهای برنامه ای کمک می کند.
\item
\lr{Kubernetes}: این ابزار یک سیستم ارکستراسیون کانتینر اپن سورس و رایگان است که اولین نسخه های آن در کمپانی گوگل طراحی شد. این ابزار به مدیریت، مقیاس بندی و به روز رسانی کانتینرهای برنامه ای بر روی یک خوشه از سرورها کمک می کند.
\item
\lr{OpenShift}: این ابزار یک پلتفرم کانتینر سازی تجاری است که بر پایه داکر و کوبرنتیس ساخته شده است. این ابزار به توسعه، اجرا و مدیریت کانتینرهای برنامه ای در محیط های ابری یا محلی کمک می کند.
\end{itemize}
کانتینرهای برنامه ای و کانتینرهای سیستمی دو نوع کانتینر هستند که بر اساس نوع برنامه های کاربردی که اجرا می کنند، تفاوت دارند. کانتینرهای برنامه ای، مثل داکر، فایل ها، وابستگی ها و کتابخانه های یک برنامه را برای اجرا در یک سیستم عامل کپسوله می کنند2. این کانتینرها فقط یک برنامه را اجرا می کنند و نیازی به یک سیستم عامل مهمان ندارند. کانتینرهای سیستمی، مثل \lr{LXC} یا \lr{LXD}، یک سیستم عامل کامل را برای اجرا چندین برنامه در یک کانتینر کپسوله می کنند. این کانتینرها مانند یک ماشین مجازی عمل می کنند، اما با استفاده از هسته سیستم عامل میزبان به جای یک هایپروایزر.

\subsection{کانتینرها چه مزایایی دارند؟}

برخی از مزایای کانتینرها عبارتند از:
\begin{itemize}[label=-]
\item
سرعت و کارایی: کانتینرها به دلیل حجم کم و استفاده بهینه از منابع سخت افزاری، سریع تر و کارآمدتر از ماشین های مجازی هستند. کانتینرها می توانند در چند ثانیه ایجاد، اجرا و حذف شوند، در حالی که ماشین های مجازی ممکن است چند دقیقه زمان ببرند.
\item
انتقال پذیری و توزیع پذیری: کانتینرها می توانند بر روی هر دستگاهی که دارای نرم افزار کانتینر سازی است، اجرا شوند. این به این معنی است که شما می توانید یک کانتینر را بر روی یک رایانه شخصی، یک سرور، یک ابر یا یک دستگاه \lr{IoT} اجرا کنید. همچنین، شما می توانید کانتینرها را به راحتی بین محیط های مختلف منتقل یا توزیع کنید.
\item
ایزولاسیون و امنیت: کانتینرها از یکدیگر و از سیستم عامل میزبان جدا هستند. این به این معنی است که اگر یک کانتینر دچار خرابی یا حمله شود، تاث
\end{itemize}

\section{روش های پیشین}

\subsection{استفاده از ماشین های مجازی}
ماشین مجازی یک نرم افزار است که به شما اجازه می دهد یک سیستم عامل دیگر را در داخل سیستم عامل فعلی خود اجرا کنید. برای استفاده از مدل های زبانی بزرگ ها، شما نیاز دارید که یک ماشین مجازی با سیستم عامل ویندوز را نصب کنید و سپس نرم افزار مدل های زبانی بزرگ را در آن اجرا کنید. این روش دارای برخی مزایا و معایب است. برخی از مزایای استفاده از ماشین مجازی عبارتند از:
\begin{itemize}[label=-]
  \item
 شما می توانید از مدل های زبانی بزرگ ها بدون نیاز به خرید یک کامپیوتر ویندوز استفاده کنید.
 \item
شما می توانید به راحتی بین سیستم عامل های مختلف جابجا شوید و فایل های خود را به اشتراک بگذارید.
\item
 شما می توانید تنظیمات و پیکربندی های مختلف را برای ماشین مجازی خود انجام دهید و در صورت لزوم به حالت قبل بازگردانید.
\end{itemize}

برخی از معایب استفاده از ماشین مجازی عبارتند از:
\begin{itemize}[label=-]
  \item
   شما نیاز دارید که فضای حافظه و پردازنده کافی را برای اجرای ماشین مجازی فراهم کنید، در غیر این صورت سرعت و عملکرد آن کند خواهد شد.
   \item
    شما نیاز دارید که یک نسخه قانونی از سیستم عامل ویندوز را تهیه و فعال کنید، در غیر این صورت با مشکلات قانونی و امنیتی روبرو خواهید شد.
    \item
     شما نمی توانید از برخی قابلیت های سخت افزاری کامپیوتر خود، مانند دوربین، صدا، چاپگر و غیره، در محیط مجازی استفاده کنید، مگر اینکه درایور های مناسب را نصب کنید.
\end{itemize}

\subsection{سرویس های ابری}
سرویس های ابری را می توان برای استفاده از مدل های زبانی بزرگ ها به عنوان یک راه حل مقیاس پذیر و انعطاف پذیر در نظر گرفت. با استفاده از سرویس های ابری، می توان از منابع محاسباتی و ذخیره سازی بدون نگرانی از محدودیت های سخت افزاری بهره برد. همچنین، می توان با استفاده از سرویس های ابری، مدل های زبانی بزرگ ها را به صورت خودکار و پویا مدیریت کرد و به روز رسانی کرد. با این حال، استفاده از سرویس های ابری نیز مشکلات خود را دارد. برخی از مشکلات عبارتند از:
\begin{itemize}[label=-]
  \item
   حفظ امنیت و حریم خصوصی داده ها و مدل های زبانی بزرگ ها در فضای ابری
   \item
   تضمین کیفیت سرویس و عملکرد مناسب مدل های زبانی بزرگ ها در شرایط نامطلوب شبکه
   \item
   هزینه بالای استفاده از سرویس های ابری برای برخی از کاربردهای مدل های زبانی بزرگ ها
   \item
    عدم وجود استانداردهای یکسان و قابل تبادل بین سرویس دهندگان مختلف ابری
\end{itemize}

\begin{table}[!ht]
  \centering
  \caption{هزینه استفاده از سرویس های ابری \lr{Open Ai}}
  \label{table:pay}
  \begin{tabular}{|l|l|l|}
  \hline
      \lr{Model} & \lr{Input} & \lr{Output} \\ \hline
      \lr{gpt-4} & \lr{\$0.03/ 1K tokens} & \lr{\$0.06/ 1K tokens} \\ \hline
      \lr{gpt-4-32k} & \lr{\$0.06/ 1K tokens} & \lr{\$0.12/ 1K tokens} \\ \hline
  \end{tabular}
\end{table}

با توجه به جدول \ref{table:pay} برای تولید 1000 توکن شما باید حدود \lr{58 \$} پرداخت کنید.

\subsection{استفاده مستقیم از مدل های زبانی بزرگ}
برای استفاده مستقیم از مدل های زبانی بزرگ نیاز مشکلات زیر را به همراه دارد
\begin{itemize}[label=-]
  \item
   نیاز به فرد متخصص
   \item
   عدم مقیاس پذیری
   \item
  امکان استفاده ان در سیستم عامل ها مختلف وجود ندارد
\end{itemize}

\section{روش پیشنهادی}
\subsection{استفاده از کانتینر ها}

کانتنر ها راهی برای بسته بندی و اجرای برنامه های کامپیوتری هستند که می توانند در محیط های مختلف اجرا شوند. کانتنر ها مزایایی مانند سادگی، قابلیت حمل و نقل، امنیت و کارایی دارند. برای انتشار مدل های زبانی بزرگ، کانتنر ها می توانند راه حل مناسبی باشند. چون:
\begin{itemize}[label=-]
  \item
   کانتنر ها می توانند مدل ها را به صورت جداگانه و مستقل از سخت افزار و سیستم عامل اجرا کنند. این به این معنی است که مدل ها را نیازی نیست برای هر پلتفرم یا دستگاه جدید تغییر داد یا تطبیق داد.
   \item
   کانتنر ها می توانند مدل ها را به صورت خودکار و پویا مقیاس بزرگ کنند. این به این معنی است که بر اساس نیاز و تقاضای کاربران، تعداد و منابع کانتنر ها را می توان افزایش یا کاهش داد.
   \item
   کانتنر ها می توانند مدل ها را به صورت امن و قابل اعتماد اجرا کنند. این به این معنی است که کانتنر ها محافظت شده از دسترسی های غیرمجاز یا خطای سخت افزار یا نرم افزار هستند.
\end{itemize}
برای استفاده از کانتنر ها برای انتشار مدل های زبانی بزرگ، لازم است چند قدم را طی کنیم:
\begin{itemize}[label=-]
  \item
   ابتدا باید یک تصویر \LTRfootnote{image} کانتنر را بسازید. تصویر کانتنر شامل کدهای، پکیج های، پیکربندی های و داده های لازم برای اجرای مدل است.
   \item
   سپس باید تصویر کانتنر را در یک رجیستر \LTRfootnote{registry} آپلود کنید. رجیستر یک سرویس ذخیره سازی است که تصویر کانتنر را در دسترس قرار می دهد.
   \item
   در نهایت باید یک نمونه \LTRfootnote{instance} از تصویر کانتنر را در یک سرویس حمل و نقل \LTRfootnote{transport} درخواست کنید. سرویس حمل و نقل چگونگی و کجای اجرای نمونه را تعیین می کند.
\end{itemize}
ولی این مراحل دسترسی سریع را برای کاربر اینجا نمی کند. چون همیچنان کار با این نمونه ساخته شده مشکل است.

\subsection{مدیریت کانتنر های ساخته شده برای مدل های زبانی بزرگ}

% !TEX root = ../ui-thesis.tex
% !TeX program = xelatex

\chapter{نتیجه‌گیری و پیشنهادها}
\section{‌نتیجه‌گیری}
`
در این مقاله، ما نشان دادیم که استفاده از کانتینر ها برای اجرای سریع و محلی مدل های زبانی بزرگ مزایای قابل توجهی دارد. کانتینر ها به ما امکان می دهند تا مدل های زبانی را بدون نیاز به نصب پیش نیاز های پیچیده و تنظیمات سخت افزاری، در هر سیستم عامل و پلتفرمی اجرا کنیم. این کار باعث افزایش قابلیت استفاده، کارایی و امنیت مدل های زبانی می شود. همچنین، کانتینر ها به ما کمک می کنند تا مدل های زبانی را به راحتی به صورت توزیع شده و مقیاس پذیر در شبکه های ابری یا لوکال اجرا کنیم. در نهایت، کانتینر ها به ما اجازه می دهند تا مدل های زبانی را با استفاده از فن آوری های جدید و بهینه سازی شده برای عملکرد بالاتر، بروز رسانی و توسعه دهیم. بنابراین، استفاده از کانتینر ها برای اجرای سریع و محلی مدل های زبانی بزرگ یک روش جذاب و قابل اعتماد است که در آینده بسیار پرکاربرد خواهد بود.
برخی دیگر از مزایای این روش به صورت زیر است:
\begin{itemize}[label=-]
  \item
با استفاده از کانتینرها، ما می‌توانیم مدل مدل ﻫﺎﻱ ﺯﺑﺎﻧﻲ ﺑﺰﺭگ  را در هر محیطی که دارای اجراکننده کانتینر باشد، به راحتی اجرا کنیم. این کانتینرها سرعت و کارایی بالایی دارند و نیاز به نصب و تنظیمات اضافی ندارند.
\item
این روش هیچ هزینه ای برای کاربران ندارد. کاربران فقط کافی است کانتینر مدل مدل ﻫﺎﻱ ﺯﺑﺎﻧﻲ ﺑﺰﺭگ  را دانلود و اجرا کنند و از آن بهره ببرند. همچنین کاربران می‌توانند کانتینر را با دیگران به اشتراک بگذارند و یا از کانتینرهای دیگران استفاده کنند.
\item
این روش امنیت داده‌های کاربر را حفظ می‌کند. کاربران نیازی ندارند که داده‌های خود را به سرورهای خارجی یا ابری ارسال کنند و یا از سرویس‌های پرداختی استفاده کنند. کانتینر مدل مدل ﻫﺎﻱ ﺯﺑﺎﻧﻲ ﺑﺰﺭگ  روی رایانه یا شبکه داخلی کاربر اجرا می‌شود و داده‌ها در دسترس کاربر می‌مانند.
\item
این روش به ما امکان می‌دهد که به یک \lr{API} واحد برای مدل های زبانی دسترسی پیدا کنیم. ما می‌توانیم از پروتوکل‌های \lr{HTTP} و \lr{gRPC} برای ارتباط با مدل مدل ﻫﺎﻱ ﺯﺑﺎﻧﻲ ﺑﺰﺭگ  استفاده کنیم و داده‌ها را با فرمت‌های مختلفی مانند \lr{JSON}، \lr{XML}، \lr{Protobuf} و غیره ارسال و دریافت کنیم. این \lr{API} واحد ما را از پیچیدگی‌های مربوط به مدل‌های زبانی مختلف مستقل می‌کند.
\end{itemize}
\section{پیشنهادها}
یکی از چالش های موجود در استفاده از مدل های زبانی بزرگ، نیاز به منابع محاسباتی زیاد و پیچیده است. این مدل ها معمولاً نیاز به تعداد زیادی از پردازنده های گرافیکی یا تنسوری دارند که همه آنها باید با هم همگام سازی شوند. این کار باعث می شود که اجرای این مدل ها در محیط های محلی یا کوچک بسیار دشوار و گران قیمت باشد. برای حل این مشکل، یک راه حل ممکن استفاده از کانتینر هاست. کانتینر ها روشی برای بسته بندی و اجرای نرم افزار ها به صورت جدا و مستقل از سخت افزار و سیستم عامل زیرین هستند. با استفاده از کانتینر ها، می توان یک محیط یکنواخت و قابل حمل برای اجرای مدل های زبانی بزرگ فراهم کرد. در این مقاله، ما پروژه ای ساختم که هدف عمده ان روی اجرای سریع محلی برای کاربر عادی یا سرور ها بود ولی مقایس پذیری این کانتنر ها در این مقاله بررسی نشده. و می توان این مورد را هم بررسی دقیق تر نمود که چگونه کانتنر های خود را بتوانیم روی چندین سرور در یک شبکه محلی قرار دهیم \cite{tamiru2020experimental}.


% ░░░░░░░▒▒▒▒▒▒▓▓▓▓ References ▓▓▓▓▒▒▒▒▒▒░░░░░░░
% References.bib نام فایل حاوی رفرنس‌ها است.
\MakeReferences{References}

% ----------------------------------------------------------------------------
%\clearpage
\pagestyle{myheadings}
%\pagenumbering{arabic}
%\setcounter{page}{1}

% ░░░░░░░▒▒▒▒▒▒▓▓▓▓ Chapters ▓▓▓▓▒▒▒▒▒▒░░░░░░░
%\clearpage
%\baselineskip=0.9cm
\linespread{1.0} % single-line spacing
%\linespread{1.3} % one and a half line spacing
%\linespread{1.6} % double-line spacing


\settextfont[Scale=1.2, ItalicFont=*, ItalicFeatures={FakeSlant=0.32}, BoldItalicFont=* Bold, BoldItalicFeatures={FakeSlant=0.32}]{B Nazanin}
\titleformat{\chapter}[display]
{\fontsize{15pt}{0}\selectfont  \flushleft  \bfseries \BNazaninScaleOne} % B Nazanin 15
{\chaptertitlename\ \tartibi{chapter}}{0.3cm}{\fontsize{15pt}{0}\selectfont \BNazaninScaleOne} % B Nazanin 15
\sectionfont{\fontsize{14pt}{\baselineskip}\selectfont \bfseries \BNazaninScaleOne} % B Nazanin 14
\subsectionfont{\fontsize{13pt}{\baselineskip}\selectfont \bfseries \BNazaninScaleOne} % B Nazanin 13
\subsubsectionfont{\fontsize{13pt}{\baselineskip}\selectfont \bfseries \BNazaninScaleOne} % B Nazanin 13

%\titlespacing*{\chapter}{0pt}{3.5cm}{8cm}
\titlespacing*{\section}{0pt}{1.2\baselineskip}{0pt}
\titlespacing*{\subsection}{0pt}{1.1\baselineskip}{0pt}
\titlespacing*{\subsubsection}{0pt}{1\baselineskip}{0pt}

\makeatletter
\renewcommand\chapter{\if@openright\cleardoublepage\else\clearpage\fi
	%\thispagestyle{empty}
	\global\@topnum\z@
	\@afterindentfalse
	\secdef\@chapter\@schapter}
\makeatother

% ░░░░░░░▒▒▒▒▒▒▓▓▓▓ Appendices ▓▓▓▓▒▒▒▒▒▒░░░░░░░
%\addtocontents{lof}{\protect\setcounter{tocdepth}{0}} % ignore adding appendix figures to list of figures
%\addtocontents{lot}{\protect\setcounter{tocdepth}{0}} % ignore adding appendix table to list of tables
\captionsetup{list=no}

\end{document} 

