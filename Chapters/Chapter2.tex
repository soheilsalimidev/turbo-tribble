% !TEX root = ../ui-thesis.tex
% !TeX program = xelatex

\chapter{مطالب اصلی}
\section{ادبیات موضوع}
\subsection{کانتینر ها}

کانتینر سازی یک روش مدیریت بسته بندی و انتشار نرم افزار است که به شکل مجازی برنامه های کاربردی را برای استقرار، بسته بندی و ایزوله می کند1. کانتینرها از هسته سیستم عامل استفاده می کنند تا برنامه های کاربردی را از سخت افزار و سیستم عامل میزبان جدا کنند. این باعث می شود که برنامه ها قابل انتقال، سبک و سریع باشند2.

مدیریت کانتینر یک فرآیند است که به ایجاد، اجرا، مانیتورینگ، توقف و حذف کانتینرها می پردازد3. برای مدیریت کانتینرها، نیاز به ابزارهایی است که به عنوان ارکستراسیون کانتینر شناخته می شوند. این ابزارها به مدیریت کانتینرهای متعدد بر روی یک یا چند سرور کمک می کنند. برخی از این ابزارها عبارتند از:
\begin{itemize}[label=-]
\item
\lr{Docker}: این ابزار یک پلتفرم کانتینر سازی است که به ساخت، اجرا و اشتراک گذاری کانتینرهای برنامه ای کمک می کند.
\item
\lr{Kubernetes}: این ابزار یک سیستم ارکستراسیون کانتینر اپن سورس و رایگان است که اولین نسخه های آن در کمپانی گوگل طراحی شد. این ابزار به مدیریت، مقیاس بندی و به روز رسانی کانتینرهای برنامه ای بر روی یک خوشه از سرورها کمک می کند.
\item
\lr{OpenShift}: این ابزار یک پلتفرم کانتینر سازی تجاری است که بر پایه داکر و کوبرنتیس ساخته شده است. این ابزار به توسعه، اجرا و مدیریت کانتینرهای برنامه ای در محیط های ابری یا محلی کمک می کند.
\end{itemize}
کانتینرهای برنامه ای و کانتینرهای سیستمی دو نوع کانتینر هستند که بر اساس نوع برنامه های کاربردی که اجرا می کنند، تفاوت دارند. کانتینرهای برنامه ای، مثل داکر، فایل ها، وابستگی ها و کتابخانه های یک برنامه را برای اجرا در یک سیستم عامل کپسوله می کنند2. این کانتینرها فقط یک برنامه را اجرا می کنند و نیازی به یک سیستم عامل مهمان ندارند. کانتینرهای سیستمی، مثل \lr{LXC} یا \lr{LXD}، یک سیستم عامل کامل را برای اجرا چندین برنامه در یک کانتینر کپسوله می کنند. این کانتینرها مانند یک ماشین مجازی عمل می کنند، اما با استفاده از هسته سیستم عامل میزبان به جای یک هایپروایزر.

\subsection{کانتینرها چه مزایایی دارند؟}

برخی از مزایای کانتینرها عبارتند از:
\begin{itemize}[label=-]
\item
سرعت و کارایی: کانتینرها به دلیل حجم کم و استفاده بهینه از منابع سخت افزاری، سریع تر و کارآمدتر از ماشین های مجازی هستند. کانتینرها می توانند در چند ثانیه ایجاد، اجرا و حذف شوند، در حالی که ماشین های مجازی ممکن است چند دقیقه زمان ببرند.
\item
انتقال پذیری و توزیع پذیری: کانتینرها می توانند بر روی هر دستگاهی که دارای نرم افزار کانتینر سازی است، اجرا شوند. این به این معنی است که شما می توانید یک کانتینر را بر روی یک رایانه شخصی، یک سرور، یک ابر یا یک دستگاه \lr{IoT} اجرا کنید. همچنین، شما می توانید کانتینرها را به راحتی بین محیط های مختلف منتقل یا توزیع کنید.
\item
ایزولاسیون و امنیت: کانتینرها از یکدیگر و از سیستم عامل میزبان جدا هستند. این به این معنی است که اگر یک کانتینر دچار خرابی یا حمله شود، تاث
\end{itemize}

\section{روش های پیشین}

\subsection{استفاده از ماشین های مجازی}
ماشین مجازی یک نرم افزار است که به شما اجازه می دهد یک سیستم عامل دیگر را در داخل سیستم عامل فعلی خود اجرا کنید. برای استفاده از مدل های زبانی بزرگ ها، شما نیاز دارید که یک ماشین مجازی با سیستم عامل ویندوز را نصب کنید و سپس نرم افزار مدل های زبانی بزرگ را در آن اجرا کنید. این روش دارای برخی مزایا و معایب است. برخی از مزایای استفاده از ماشین مجازی عبارتند از:
\begin{itemize}[label=-]
  \item
 شما می توانید از مدل های زبانی بزرگ ها بدون نیاز به خرید یک کامپیوتر ویندوز استفاده کنید.
 \item
شما می توانید به راحتی بین سیستم عامل های مختلف جابجا شوید و فایل های خود را به اشتراک بگذارید.
\item
 شما می توانید تنظیمات و پیکربندی های مختلف را برای ماشین مجازی خود انجام دهید و در صورت لزوم به حالت قبل بازگردانید.
\end{itemize}

برخی از معایب استفاده از ماشین مجازی عبارتند از:
\begin{itemize}[label=-]
  \item
   شما نیاز دارید که فضای حافظه و پردازنده کافی را برای اجرای ماشین مجازی فراهم کنید، در غیر این صورت سرعت و عملکرد آن کند خواهد شد.
   \item
    شما نیاز دارید که یک نسخه قانونی از سیستم عامل ویندوز را تهیه و فعال کنید، در غیر این صورت با مشکلات قانونی و امنیتی روبرو خواهید شد.
    \item
     شما نمی توانید از برخی قابلیت های سخت افزاری کامپیوتر خود، مانند دوربین، صدا، چاپگر و غیره، در محیط مجازی استفاده کنید، مگر اینکه درایور های مناسب را نصب کنید.
\end{itemize}

\subsection{سرویس های ابری}
سرویس های ابری را می توان برای استفاده از مدل های زبانی بزرگ ها به عنوان یک راه حل مقیاس پذیر و انعطاف پذیر در نظر گرفت. با استفاده از سرویس های ابری، می توان از منابع محاسباتی و ذخیره سازی بدون نگرانی از محدودیت های سخت افزاری بهره برد. همچنین، می توان با استفاده از سرویس های ابری، مدل های زبانی بزرگ ها را به صورت خودکار و پویا مدیریت کرد و به روز رسانی کرد. با این حال، استفاده از سرویس های ابری نیز مشکلات خود را دارد. برخی از مشکلات عبارتند از:
\begin{itemize}[label=-]
  \item
   حفظ امنیت و حریم خصوصی داده ها و مدل های زبانی بزرگ ها در فضای ابری
   \item
   تضمین کیفیت سرویس و عملکرد مناسب مدل های زبانی بزرگ ها در شرایط نامطلوب شبکه
   \item
   هزینه بالای استفاده از سرویس های ابری برای برخی از کاربردهای مدل های زبانی بزرگ ها
   \item
    عدم وجود استانداردهای یکسان و قابل تبادل بین سرویس دهندگان مختلف ابری
\end{itemize}

\begin{table}[!ht]
  \centering
  \caption{هزینه استفاده از سرویس های ابری \lr{Open Ai}}
  \label{table:pay}
  \begin{tabular}{|l|l|l|}
  \hline
      \lr{Model} & \lr{Input} & \lr{Output} \\ \hline
      \lr{gpt-4} & \lr{\$0.03/ 1K tokens} & \lr{\$0.06/ 1K tokens} \\ \hline
      \lr{gpt-4-32k} & \lr{\$0.06/ 1K tokens} & \lr{\$0.12/ 1K tokens} \\ \hline
  \end{tabular}
\end{table}

با توجه به جدول \ref{table:pay} برای تولید 1000 توکن شما باید حدود \lr{58 \$} پرداخت کنید.

\subsection{استفاده مستقیم از مدل های زبانی بزرگ}
برای استفاده مستقیم از مدل های زبانی بزرگ نیاز مشکلات زیر را به همراه دارد
\begin{itemize}[label=-]
  \item
   نیاز به فرد متخصص
   \item
   عدم مقیاس پذیری
   \item
  امکان استفاده ان در سیستم عامل ها مختلف وجود ندارد
\end{itemize}

\section{روش پیشنهادی}
\subsection{استفاده از کانتینر ها}

کانتنر ها راهی برای بسته بندی و اجرای برنامه های کامپیوتری هستند که می توانند در محیط های مختلف اجرا شوند. کانتنر ها مزایایی مانند سادگی، قابلیت حمل و نقل، امنیت و کارایی دارند. برای انتشار مدل های زبانی بزرگ، کانتنر ها می توانند راه حل مناسبی باشند. چون:
\begin{itemize}[label=-]
  \item
   کانتنر ها می توانند مدل ها را به صورت جداگانه و مستقل از سخت افزار و سیستم عامل اجرا کنند. این به این معنی است که مدل ها را نیازی نیست برای هر پلتفرم یا دستگاه جدید تغییر داد یا تطبیق داد.
   \item
   کانتنر ها می توانند مدل ها را به صورت خودکار و پویا مقیاس بزرگ کنند. این به این معنی است که بر اساس نیاز و تقاضای کاربران، تعداد و منابع کانتنر ها را می توان افزایش یا کاهش داد.
   \item
   کانتنر ها می توانند مدل ها را به صورت امن و قابل اعتماد اجرا کنند. این به این معنی است که کانتنر ها محافظت شده از دسترسی های غیرمجاز یا خطای سخت افزار یا نرم افزار هستند.
\end{itemize}
برای استفاده از کانتنر ها برای انتشار مدل های زبانی بزرگ، لازم است چند قدم را طی کنیم:
\begin{itemize}[label=-]
  \item
   ابتدا باید یک تصویر \LTRfootnote{image} کانتنر را بسازید. تصویر کانتنر شامل کدهای، پکیج های، پیکربندی های و داده های لازم برای اجرای مدل است.
   \item
   سپس باید تصویر کانتنر را در یک رجیستر \LTRfootnote{registry} آپلود کنید. رجیستر یک سرویس ذخیره سازی است که تصویر کانتنر را در دسترس قرار می دهد.
   \item
   در نهایت باید یک نمونه \LTRfootnote{instance} از تصویر کانتنر را در یک سرویس حمل و نقل \LTRfootnote{transport} درخواست کنید. سرویس حمل و نقل چگونگی و کجای اجرای نمونه را تعیین می کند.
\end{itemize}
ولی این مراحل دسترسی سریع را برای کاربر اینجا نمی کند. چون همیچنان کار با این نمونه ساخته شده مشکل است.

\subsection{مدیریت کانتنر های ساخته شده برای مدل های زبانی بزرگ}
