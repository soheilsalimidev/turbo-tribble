% !TEX root = ../ui-thesis.tex
% !TeX program = xelatex

\chapter{نتیجه‌گیری و پیشنهادها}
\section{‌نتیجه‌گیری}
`
در این مقاله، ما نشان دادیم که استفاده از کانتینر ها برای اجرای سریع و محلی مدل های زبانی بزرگ مزایای قابل توجهی دارد. کانتینر ها به ما امکان می دهند تا مدل های زبانی را بدون نیاز به نصب پیش نیاز های پیچیده و تنظیمات سخت افزاری، در هر سیستم عامل و پلتفرمی اجرا کنیم. این کار باعث افزایش قابلیت استفاده، کارایی و امنیت مدل های زبانی می شود. همچنین، کانتینر ها به ما کمک می کنند تا مدل های زبانی را به راحتی به صورت توزیع شده و مقیاس پذیر در شبکه های ابری یا لوکال اجرا کنیم. در نهایت، کانتینر ها به ما اجازه می دهند تا مدل های زبانی را با استفاده از فن آوری های جدید و بهینه سازی شده برای عملکرد بالاتر، بروز رسانی و توسعه دهیم. بنابراین، استفاده از کانتینر ها برای اجرای سریع و محلی مدل های زبانی بزرگ یک روش جذاب و قابل اعتماد است که در آینده بسیار پرکاربرد خواهد بود.
برخی دیگر از مزایای این روش به صورت زیر است:
\begin{itemize}[label=-]
  \item
با استفاده از کانتینرها، ما می‌توانیم مدل مدل ﻫﺎﻱ ﺯﺑﺎﻧﻲ ﺑﺰﺭگ  را در هر محیطی که دارای اجراکننده کانتینر باشد، به راحتی اجرا کنیم. این کانتینرها سرعت و کارایی بالایی دارند و نیاز به نصب و تنظیمات اضافی ندارند.
\item
این روش هیچ هزینه ای برای کاربران ندارد. کاربران فقط کافی است کانتینر مدل مدل ﻫﺎﻱ ﺯﺑﺎﻧﻲ ﺑﺰﺭگ  را دانلود و اجرا کنند و از آن بهره ببرند. همچنین کاربران می‌توانند کانتینر را با دیگران به اشتراک بگذارند و یا از کانتینرهای دیگران استفاده کنند.
\item
این روش امنیت داده‌های کاربر را حفظ می‌کند. کاربران نیازی ندارند که داده‌های خود را به سرورهای خارجی یا ابری ارسال کنند و یا از سرویس‌های پرداختی استفاده کنند. کانتینر مدل مدل ﻫﺎﻱ ﺯﺑﺎﻧﻲ ﺑﺰﺭگ  روی رایانه یا شبکه داخلی کاربر اجرا می‌شود و داده‌ها در دسترس کاربر می‌مانند.
\item
این روش به ما امکان می‌دهد که به یک \lr{API} واحد برای مدل های زبانی دسترسی پیدا کنیم. ما می‌توانیم از پروتوکل‌های \lr{HTTP} و \lr{gRPC} برای ارتباط با مدل مدل ﻫﺎﻱ ﺯﺑﺎﻧﻲ ﺑﺰﺭگ  استفاده کنیم و داده‌ها را با فرمت‌های مختلفی مانند \lr{JSON}، \lr{XML}، \lr{Protobuf} و غیره ارسال و دریافت کنیم. این \lr{API} واحد ما را از پیچیدگی‌های مربوط به مدل‌های زبانی مختلف مستقل می‌کند.
\end{itemize}
\section{پیشنهادها}
یکی از چالش های موجود در استفاده از مدل های زبانی بزرگ، نیاز به منابع محاسباتی زیاد و پیچیده است. این مدل ها معمولاً نیاز به تعداد زیادی از پردازنده های گرافیکی یا تنسوری دارند که همه آنها باید با هم همگام سازی شوند. این کار باعث می شود که اجرای این مدل ها در محیط های محلی یا کوچک بسیار دشوار و گران قیمت باشد. برای حل این مشکل، یک راه حل ممکن استفاده از کانتینر هاست. کانتینر ها روشی برای بسته بندی و اجرای نرم افزار ها به صورت جدا و مستقل از سخت افزار و سیستم عامل زیرین هستند. با استفاده از کانتینر ها، می توان یک محیط یکنواخت و قابل حمل برای اجرای مدل های زبانی بزرگ فراهم کرد. در این مقاله، ما پروژه ای ساختم که هدف عمده ان روی اجرای سریع محلی برای کاربر عادی یا سرور ها بود ولی مقایس پذیری این کانتنر ها در این مقاله بررسی نشده. و می توان این مورد را هم بررسی دقیق تر نمود که چگونه کانتنر های خود را بتوانیم روی چندین سرور در یک شبکه محلی قرار دهیم \cite{tamiru2020experimental}.